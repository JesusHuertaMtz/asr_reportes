El analizar las redes de computadoras, sus componentes y tráfico en éstas es de vital importancia en las empresas, organizaciones, escuelas, etc. pues, nos permiten conocer, entre otras cosas, el estado de las mismas y su comportamiento ante determinadas situaciones o acciones. Por otro lado, nos permite también, predecir aspectos de nuestros recursos en la red, para prevenir o advertir determinadas circunstancias no deseadas en el comportamiento de la red, todo por medio de técnicas y procesos aprendidos en clase, como lo son línea de base, mínimos cuadrados o suavizamiento exponencial. Así, podemos tener un mayor conocimiento y control sobre nuestras redes y lo que a éstas les pueda ocurrir. 