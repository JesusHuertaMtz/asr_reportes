\documentclass[oneside,10pt]{book}

\usepackage{cdtBook}
\usepackage{usecases}

\title{Reporte Problema 2}
\subtitle{Administración de Servicios en Red}
\author{Huerta Martínez Jesús Manuel, Monteón Valdéz Raúl Kevin, Olivares García Marco Antonio.}
%\organization{Escuela Superior de Cómputo, IPN}


%%%%%%%%%%%%%%%%%%%%%%%%%%%%%%%%%%%%%%%%%%%%%%%%%%%%%%%%%%%%%%%%
\begin{document}

\maketitle
\thispagestyle{empty}

\frontmatter
\tableofcontents

\mainmatter

% =================================================================
%					               INTRODUCCIÓN
% =================================================================

\chapter{Introducción.}

\cfinput{introduccion/introduccion}

% =================================================================
%					                 DESARROLLO
% =================================================================

\chapter{Desarrollo de la práctica.}

\cfinput{desarrollo/desarrollo}

% =================================================================
%							PARTE 1
% =================================================================

\section{Parte 1 - Análisis y ajuste del rendimiento (línea de base).}

% Tarea 1.
\subsection{Tarea 1: Inventario de la configuración.}
\cfinput{desarrollo/partes/parte1/parte1_tarea1}

% Tarea 2.
\subsection{Tarea 2: Verificar que SNMP MIB se admita en el host.}
\cfinput{desarrollo/partes/parte1/parte1_tarea2}

% Tarea 3.
\subsection{Tarea 3: Consultar y registrar objetos MIB del SNMP específicos del HOST.}
\cfinput{desarrollo/partes/parte1/parte1_tarea3}

% Tarea 4.
\subsection{Tarea 4: Analice los datos para determinar los umbrales.}
\cfinput{desarrollo/partes/parte1/parte1_tarea4}

% Tarea 5.
\subsection{Tarea 5: Problemas inmediatos identificados arreglo.}
\cfinput{desarrollo/partes/parte1/parte1_tarea5}

% =================================================================
%							PARTE 2
% =================================================================

\section{Parte 2 - Detección de comportamiento anómalo al monitorizar una red.}

% Tarea 1.
\subsection{Tarea 1: Predicción de la tendencia de series temporales lineales.}
\cfinput{desarrollo/partes/parte2/parte2_tarea1}

% Tarea 2.
\subsection{Tarea 2: Predicción de la tendencia de series temporales no linealest.}
\cfinput{desarrollo/partes/parte2/parte2_tarea2}

% =================================================================
%						     CONCLUSIONES
% =================================================================

\chapter{Conclusiones.}

% Jesús.
\section{Huerta Martínez Jesús Manuel.}
\cfinput{conclusiones/jesus}

% Kevin.
\section{Monteón Valdéz Raúl Kevin.}
\cfinput{conclusiones/kevin}

% Marco.
\section{Olivares García Marco Antoniol.}
\cfinput{conclusiones/marco}

% =================================================================
%						     BIBLIOGRAFÍA
% =================================================================

\chapter{Bibliografía.}

\cfinput{bibliografia/bibliografia}
	
\end{document}

