\noindent
Una red de computadoras (también llamada red informática) es un conjunto equipos (computadoras y dispositivos), conectados por medio de cables, señales, ondas o cualquier otro método de transporte de datos, para compartir información (archivos), recursos (discos, impresoras, programas, etc.) y servicios (acceso a una base de datos, internet, correo electrónico, chat, juegos, etc.). A
cada una de las computadoras conectadas a la red se le denomina un nodo. \cite{def_redComputadoras}

\noindent
\newline
Hoy en día, las redes de computadoras, se han vuelto fundamentales en las empresas, pues se les han encontrado numerosos aspectos que facilitan los procesos y la comunicación dentro de la misma. 
\newline
Hay al menos 3 ventajas básicas por las que te beneficiaría disponer de una buena red de ordenadores en un entorno empresarial. No nos referimos sólo a la conexión a Internet, sino al uso de otros dispositivos.
\begin{itemize}
	\item Podrás compartir los recursos informáticos. Supongamos que en una empresa trabajan 20 o 30 personas en las que hay idéntico número de computadoras. Si instalamos una red local, todos podrían transmitir los archivos con los que están trabajando en su PC, incluso si se encuentran en dos edificios distintos (en un radio cercano). Esto quiere decir que puede compartirse el software y los archivos comerciales de una manera mucho más rápida y eficiente.
	\item Tendrás más velocidad de transmisión de datos. Cuando conectamos las computadoras en red, comparten también su capacidad de transmisión de datos, de manera que la gestión de las tareas se vuelve mucho más ágil y rápida, con el ahorro de tiempo y esfuerzo que esto supone. Una red puede tener velocidad desde 10 Mbps hasta 1 Gbps.
	\item 3. Ahorrarás en hardware, software y espacio. No es necesario disponer, por ejemplo, de decenas de impresoras en la planta de un mismo edificio. Basta con que haya una central que esté conectada en red y todos podrán imprimir en la misma. Además podrás usar software de red. Esto supone un importante ahorro de dinero, pero también permite disponer sólo de los dispositivos necesarios. \cite{ventajas_redEmpresa}
\end{itemize}

\noindent
Así, nos queda claro que es de suma importancia el hecho de tener una red, pues nos da acceso a muchas ventajas. Sin embargo, las redes pueden ser volátiles y poco seguras si no saben tratarse. Para ello, recurrimos a la administración de la propia red. 
\newline
El modelo de administración de red de la Organización internacional para la normalización (ISO) define cinco áreas funcionales de la administración de red. Este documento abarca todas las áreas funcionales. El objetivo general de este documento es ofrecer recomendaciones prácticas en cada área funcional para aumentar la efectividad total de las prácticas y herramientas de administración actuales. También proporciona pautas de diseño para la implementación futura de tecnologías y herramientas de administración de redes.
\newline
Las cinco áreas funcionales del modelo de administración de red ISO son mencionadas abajo.
\begin{itemize}
	\item Administración de fallas — Detecte, aísle, notifique, y corrija los incidentes encontrados en la red.
	\item Administración de la configuración — Configuraciones de aspecto de los dispositivos de red tales como administración de archivos de configuración, Administración de inventario, y administración del software.
	\item Administración del rendimiento — Monitoree y mida los diversos aspectos del rendimiento para poder mantener el rendimiento general en un nivel aceptable.
	\item Administración de seguridad — Proporcione el acceso a los dispositivos de red y a los recursos corporativos a los individuos autorizados.
	\item Administración de contabilidad — Información de uso de un recurso de la red.
\end{itemize}

\noindent
Una de las partes fundamentale para que una empresa conozco y tenga control de su red y dispositivos conectados es la ya mencionada \textbf{Administración del rendimiento}, misma que se explica a continuación.

\section{Administración del rendimiento.}

\subsection{Contrato de nivel de servicio.}

\noindent
Un acuerdo de nivel de servicio (SLA) es un acuerdo escrito entre el proveedor del servicio y sus clientes sobre el nivel de rendimiento esperado de los servicios de red. SLA consiste en la métrica convenida en entre el proveedor y sus clientes. Los valores configurados para las mediciones deben ser realistas, significativos y cuantificables para ambas partes.
\newline
\newline
Las diversas estadísticas de la interfaz se pueden recoger de los dispositivos de red para medir el nivel de rendimiento. Estas estadísticas pueden incluirse como métricas en el SLA. Las estadísticas tales como caídas de entradas en la cola, pérdidas de la cola de salida, y paquetes ignorados son útiles para diagnosticar los problemas relacionados con el rendimiento.
\newline
\newline
A nivel de dispositivos, la medición del rendimiento pude incluir el uso de la CPU, la asignación del búfer (búfer grande y mediano, fallas y radio hit) y la asignación de la memoria. El rendimiento de ciertos protocolos de red está directamente relacionado con la disponibilidad de memoria intermedia en los dispositivos de red. La medición de las estadísticas de rendimiento a nivel del dispositivo son fundamentales para optimizar el rendimiento de los protocolos de alto nivel.
\newline
\newline
Los dispositivos de red tales como Routers soportan los diversos protocolos de capa más altas tales como Data Link Switching Workgroup (DLSW), el Remote Source Route Bridging (RSRB), APPLETALK, y así sucesivamente. Pueden controlarse y recolectarse las estadísticas de rendimiento de las tecnologías de Wide Area Network (WAN) incluyendo Frame Relay, ATM, el Integrated Services Digital Network (ISDN) y otras. \cite{admin_redCisco}
\newline

\subsection{Supervisión del rendimiento, medición e informes}

\noindent
Las diferentes mediciones del rendimiento en la interfaz, en el dispositivo o en los niveles de protocolo deberían recolectarse de forma regular utilizando SNMP. El motor del sondeo en un sistema de administración de red puede ser utilizado para fines de recolección de datos. La mayoría de los sistemas de administración de red son capaces de recolectar, almacenar y presentar los datos de sondeo.
\newline
\newline
Las diversas soluciones están disponibles en el mercado dirigir las necesidades de la Administración del rendimiento de los entornos para empresas. Estos sistemas son capaces de recolectar, almacenar y presentar datos de los dispositivos de red y los servidores. El interfaz basada en la Web en la mayoría de los Productos hace los Datos del rendimiento accesibles dondequiera adentro de la empresa. Una evaluación de los productos antes mencionados determinará si cumplen con los requisitos de diferentes usuarios. \cite{admin_redCisco}
\newline 

\section{Linea de Base.}

\noindent
Una línea de fondo es un proceso para estudiar la red a intervalos regulares para asegurarse de que la red está trabajando según lo diseñado. Es más que un solo informe que detalla la salud de la red en cierta punta a tiempo. Siguiendo el proceso de línea de base, usted puede obtener la siguiente información:
\begin{itemize}
	\item Gane la información valiosa en la salud del hardware y software
	\item Determine los usos de recurso de la red actuales
	\item Tome las decisiones precisas sobre los umbrales de la alarma de red
	\item Identifique los problemas de la red actual
	\item Prediga los problemas futuros
\end{itemize}

\noindent
El propósito de una línea de fondo es determinar: 
\begin{itemize}
	\item Donde está su red en la línea verde
	\item Cómo rápidamente la carga de la red está aumentando
	\item Esperanzadamente prediga en qué punta a tiempo entrecruzarán los dos
\end{itemize}

\noindent
Realizando una línea de fondo en las bases normales, usted puede descubrir al estado actual y extrapolar cuando los errores ocurrirán y se
prepararán para ellos por adelantado. Esto también lo ayuda a tomar decisiones más informadas sobre cuándo, dónde y cómo gastar dinero del
presupuesto en actualizaciones de la red. \cite{lineaBase}

\section{Método de mínimos cuadrados.}

\noindent
Una línea del mejor ajuste es una recta que muestra la mejor aproximación del conjunto dado de datos dispersos. Se utiliza para estudiar la naturaleza de la relación entre dos variables.
\newline
Una línea del mejor ajuste puede determinarse usando un método de “simple vista” dibujando una línea recta en un diagrama de dispersión de modo que el número de puntos por encima y por debajo de la línea sea aproximadamente igual.
\newline
Una forma más precisa de encontrar la línea del mejor ajuste es el método de mínimos cuadrdados.
\newline
\newline
Los siguientes pasos sirven para encontrar la ecuación lineal del mejor ajuste para un conjunto de pares ordenados (x1, y1), (x2, y2), ..., (xn, yn).
\begin{itemize}
	\item Calcular la media de los valores x y la media de los valores y.
	\item Obtener la pendiente de la línea de mejor ajuste.
	\item Calcule la intercepción en y de la línea.
	\item Utilice la pendiente m y la intercepción en y b para formar la ecuación de la recta.
\end{itemize}

\section{Suavizamiento exponencial (Holt Winters).}

\noindent
Algunas de las técnicas incluidas en la familia de series temporales conocida como alisado o suavizamiento exponencial pueden extrapolarse a entornos de negocio altamente competitivos. Holt-Winters y Box-Jenkins son dos de las más relevantes. Sin embargo, el modelo de series temporales Holt-Winters resulta especialmente útil para realizar análisis y pronósticos de negocio, debido a su facilidad de uso y a sus resultados inmediatos.
\newline
\newline
El modelo Holt-Winters incorpora un conjunto de procedimientos que conforman el núcleo de la familia de series temporales de suavizamiento exponencial. A diferencia de muchas otras técnicas, el modelo Holt-Winters puede adaptarse fácilmente a cambios y tendencias, así como a patrones estacionales. En comparación con otras técnicas, como ARIMA, el tiempo necesario para calcular el pronóstico es considerablemente más rápido. Esto significa que cualquier usuario – con suficiente pero no necesariamente mucha experiencia – puede poner en práctica la técnica de Holt-Winters. Más allá de sus características técnicas, su aplicación en entornos de negocio es muy común. De hecho, Holt-Winters se utiliza habitualmente por muchas compañías para pronosticar la demanda a corto plazo cuando los datos de venta contienen tendencias y patrones estacionales de un modo subyacente.
\newline
\newline
Según Goodwin, Paul, los pronósticos de ventas mensuales requieren tres componentes para realizar la ecuación:
\begin{itemize}
	\item El actual nivel de ventas subyacente, que permanece tras haber desestacionalizado las ventas y haber restado el efecto de factores aleatorios.
	\item La tendencia actual que siguen las ventas. Es decir, el cambio en el nivel subyacente de ventas que esperamos suceda entre el momento actual y el próximo mes. Por ejemplo, si estimamos nuestro nivel actual en 500 unidades y esperamos que sea de 505 unidades el siguiente mes, entonces nuestra tendencia estimada es de +5 unidades.
	\item El índice estacional para el mes que estamos pronosticando. Si nuestra estimación es 1.2, esto significa que esperamos que nuestras ventas este mes sean 20\% por encima del nivel subyacente de dicho mes, mostrando así que nuestros productos se venden relativamente bien en ese momento del año.
\end{itemize}