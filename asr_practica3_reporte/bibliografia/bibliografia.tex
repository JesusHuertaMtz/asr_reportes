
\begin{thebibliography}{X}

	\bibitem{def_redComputadoras} 
		\textsc{CM Mansilla,}
		\textit{Redes de Computadoras} [En línea].
		Universidad Nacional de Litoral. Argentina, (S/F). 
		\newline
		Disponible en 
		\url{http://www.fca.unl.edu.ar/informaticabasica/Redes.pdf}

	\bibitem{ventajas_redEmpresa} 
		\textsc{S. Juliá,}
		\textit{Ventajas de tener una red de ordenadores en la empresa} [En línea].
		GADAE. (S/L), (S/F). 
		\newline
		Disponible en 
		\url{http://www.gadae.com/blog/ventajas-red-de-ordenadores-empresa/}

	\bibitem{estatico} 
		\textsc{Universidad Veracruzana,}
		\textit{Enrutamiento estático} [En línea].
		Facultad de estadística e informática, Universidad Veracruzana. México, (S/F). 
		\newline
		Disponible en 
		\url{https://www.uv.mx/personal/ocruz/files/2014/01/Enrutamiento-estatico.pdf}

	\bibitem{rip} 
		\textsc{E. Duarte,}
		\textit{Cisco CCNA – Cómo Configurar Protocolo RIP En Cisco Router} [En línea].
		Capacity: Information Technology Academy. (S/L), 2014. 
		\newline
		Disponible en 
		\url{http://blog.capacityacademy.com/2014/06/20/cisco-ccna-como-configurar-protocolo-rip-en-cisco-router/}
		
	\bibitem{ospf} 
		\textsc{E. Duarte,}
		\textit{Cisco CCNA – Cómo Configurar OSPF En Cisco Router} [En línea].
		Capacity: Information Technology Academy. (S/L), 2014. 
		\newline
		Disponible en 
		\url{http://blog.capacityacademy.com/2014/06/23/cisco-ccna-como-configurar-ospf-en-cisco-router/}
\end{thebibliography}

