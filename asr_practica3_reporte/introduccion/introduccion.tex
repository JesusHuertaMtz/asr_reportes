\noindent
Una red de computadoras (también llamada red informática) es un conjunto equipos (computadoras y dispositivos), conectados por medio de cables, señales, ondas o cualquier otro método de transporte de datos, para compartir información (archivos), recursos (discos, impresoras, programas, etc.) y servicios (acceso a una base de datos, internet, correo electrónico, chat, juegos, etc.). A
cada una de las computadoras conectadas a la red se le denomina un nodo. \cite{def_redComputadoras}

\noindent
\newline
Hoy en día, las redes de computadoras, se han vuelto fundamentales en las empresas, pues se les han encontrado numerosos aspectos que facilitan los procesos y la comunicación dentro de la misma. 
\newline
Hay al menos 3 ventajas básicas por las que te beneficiaría disponer de una buena red de ordenadores en un entorno empresarial. No nos referimos sólo a la conexión a Internet, sino al uso de otros dispositivos.
\begin{itemize}
	\item Podrás compartir los recursos informáticos. Supongamos que en una empresa trabajan 20 o 30 personas en las que hay idéntico número de computadoras. Si instalamos una red local, todos podrían transmitir los archivos con los que están trabajando en su PC, incluso si se encuentran en dos edificios distintos (en un radio cercano). Esto quiere decir que puede compartirse el software y los archivos comerciales de una manera mucho más rápida y eficiente.
	\item Tendrás más velocidad de transmisión de datos. Cuando conectamos las computadoras en red, comparten también su capacidad de transmisión de datos, de manera que la gestión de las tareas se vuelve mucho más ágil y rápida, con el ahorro de tiempo y esfuerzo que esto supone. Una red puede tener velocidad desde 10 Mbps hasta 1 Gbps.
	\item 3. Ahorrarás en hardware, software y espacio. No es necesario disponer, por ejemplo, de decenas de impresoras en la planta de un mismo edificio. Basta con que haya una central que esté conectada en red y todos podrán imprimir en la misma. Además podrás usar software de red. Esto supone un importante ahorro de dinero, pero también permite disponer sólo de los dispositivos necesarios. \cite{ventajas_redEmpresa}
\end{itemize}

\noindent 
En el presente documento se tratan las diferentes formas para establecer una red y la forma en que se logra comunicar la misma por medio de protocolos de enrutamiento y e describen ejemplos de sensores que pueden ser útiles en las redes para conocer su estado y la información que en éstas se comparte.
